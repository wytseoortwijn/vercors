\chapter{Structure}

The VerCors sources have been divided into several modules.
These modules are separate because they track external
projects and or may be reusable in different contexts.

First, we will give an overview of the modules.
Second, we will describe how to configure Eclipse to develop
(parts of) the VerCors tool set.

\section{Modules that are part of the VerCors Tool}

\begin{description}
\item[HRE] The Hybrid Runtime Environment contains the classes that can potentially be
reused by projects that have nothing to do with program verification. For example,
the classes that are used for the following tasks are located in HRE:
\begin{itemize}
\item Running command line tools.
\item Text output (logging) management.
\item Generic debug utilities.
\item Command line parsing and configurations.
\item Plugin infra-structure based on class loaders.
\end{itemize}
\item[Viper API]
This project contains the Java API that is used to access the functionality of the Viper project.
\item[core] This is where most of the implementation of VerCors lives:
\begin{itemize}
\item parsers/pretty printers
\item program transformers
\item backends
\end{itemize}
\item[main] For historic reasons the main program had to be in a separate project.
Thus in this prohect, one may find:
\begin{itemize}
\item The VerCors main program.
\item The command line test infrastructure.
\end{itemize}
The first could be moved into the core project. The second could be refactored into
parts that belong in the HRE projects and parts that belong in core.
\end{description}

\section{Modules that are part of the VerCors Tool Viper plugins.}

\begin{description}
\item[Viper API]
This project contains the Java API that is used to access the functionality of the Viper project.
Note that an instance of a plugin will only work for an instance of a tool if both have been
linked using the same version of the API.
\item[Silver]
This project tracks the Viper silver module, which contains the Sivler parser and AST.
It has been extended with implementations of the Viper API to allow interfacing with
the VerCors Tool.
\item[Carbon]
This project tracks the Viper Carbon module, which implements the Carbon verifier.
It has been extended with a basic implementation of the Viper API.
\item[Silicon]
This project tracks the Viper Silicon module, which implements the Silicon verifier.
It has been extended with an advanced implementation of the Viper API.
\end{description}

\section{Setting up for developping with Eclipse.}

In order to avoid having to specify the entire list of dependencies of the Viper project,
we simply use the jars assembled for Carbon and Silicon to provide the dependencies.
Thus, the first step of setting up is to perform a command line build.
Also, if you are going to develop any of the Viper modules, you will
need to install the Scala IDE plugin for Eclispe.
Afterwards, several sub-directories are used as the (non-standard) locations of eclipse projects.


\begin{enumerate}
\item Install the Scala IDE plugin for Eclipse.
\item Perform a command line build:
\begin{verbatim}
> ant
\end{verbatim}
\item Create a Java project HRE with the sub-directory hre as its root.
\item Create a Java project Viper API with the sub-directory viper/viper-api as its root.
\item Create a Java project VCT core with the sub-directory core as its root.
\\
Add the following dependencies:
\begin{itemize}
\item project HRE
\item project Viper API
\end{itemize}
\item Create a Java project VCT main with the sub-directory main as its root.
\\
Add the following dependencies:
\begin{itemize}
\item project HRE
\item project VCT core
\end{itemize}
\item Create a Scala project Silver with the sub-directory viper/silver as its root.
\\
Further configuration is needed, but will be performed in a later step.
\item Create a Scala project Silicon with the sub-directory viper/silicon as its root.
\\
Further configuration is needed, but will be performed in a later step.
\item Create a Scala project Carbon with the sub-directory viper/carbon as its root.
\\
Further configuration is needed, but will be performed in a later step.
\item Fix the setup for Silver:
\begin{verbatim}
sources:
  silver/src/main/scala

output:
  Silver/target/scala-2.11/classes

dependencies:
  project Viper API
  Silicon/target/scala-2.11/silicon.jar
  Java
\end{verbatim}
\item Fix setup for Silicon:
\begin{verbatim}
sources:
  silicon/src/main/scala

output:
  Silicon/target/scala-2.11/classes

dependencies:
  project Viper API
  Silicon/target/scala-2.11/silicon.jar
  Java
\end{verbatim}
\item Fix setup for Carbon:
\begin{verbatim}
sources:
  carbon/src/main/scala

output:
  Carbon/target/scala-2.11/classes

dependencies:
  project Viper API
  Carbon/target/scala-2.11/carbon.jar
  Java
\end{verbatim}
\end{enumerate}



