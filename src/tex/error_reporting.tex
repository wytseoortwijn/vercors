\chapter{Error Reporting}

The error reporting mechanism is built around the notion of an \emph{origin}.
Every transformation maintains the origin of every AST node that it produces.
This origin contains enough information to both trace back every AST node
to the source code from which it was generated and to convert error
messages with respect to the output to the correct message for the input.
E.g. an assertion failed in the output, might be reported as a
post-condition failure at a return statement.


\section{Implementation}

The class \lstinline+vct.logging.VerCorsError+ captures the essence of the
VerCors error model. The ErrorCode, which tells what was wrong and the
SubCode saying why. Many of the encodings of the VerCors toolset
encode constructs and/or proof obligations in various ways.

This means that the list of errors with respect to the output of every pass
has to be translated to be a list of error with respect to the input of the
pass. The simplest ways of implementing this, is to call the
method \lstinline+set_branch+ on generated code for a certain proof obligation
and, whose origin has to be the location where the error should be reported.
In addition an \lstinline+vct.logging.ErrorMapping+ has to be maintained
which performs a simple lookup per branch of errors.

For example, when a return statement is converted to a sequence
of an assertion of the post-condition and then an assume false,
the assertion failed error is translated to a postcondition failed.


